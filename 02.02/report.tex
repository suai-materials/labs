% Created 2023-03-09 Thu 12:09
% Intended LaTeX compiler: pdflatex
\documentclass[a4paper,14pt]{extarticle}
\usepackage[utf8]{inputenc}
\usepackage[T1]{fontenc}
\usepackage{graphicx}
\usepackage{longtable}
\usepackage{wrapfig}
\usepackage{rotating}
\usepackage[normalem]{ulem}
\usepackage{amsmath}
\usepackage{amssymb}
\usepackage{capt-of}
\usepackage{hyperref}
\usepackage{minted}
\usepackage[utf8x]{inputenc}
\usepackage[T2A]{fontenc}
\usepackage[russian]{babel}
\usepackage{tempora}
\usepackage{geometry}
\geometry{a4paper, left=30mm, top=20mm, bottom=20mm, right=15mm }
\usepackage{graphicx}
\usepackage{array}
\usepackage{tabularx}
\usepackage{listings}
\usepackage{float}
\usepackage{setspace}
\usepackage{tabularx}
\usepackage{longtable}
\usepackage{titlesec}
\titleformat*{\section}{\large\bfseries}
\titleformat*{\subsection}{\normalsize\bfseries}
\titleformat*{\subsubsection}{\normalsize\bfseries}
\addto\captionsrussian{\renewcommand{\contentsname}{\centering \normalsize СОДЕРЖАНИЕ}}
\addtocontents{toc}{\protect\thispagestyle{empty}}
\usepackage{titletoc}
\titlecontents{section}[0pt]{}{\contentsmargin{0pt} \thecontentslabel\enspace}{\contentsmargin{0pt}}{\titlerule*[0.5pc]{.}\contentspage}[]
\dottedcontents{subsection}[3.1em]{}{1.5em}{0.5pc}
\usepackage{caption}
\DeclareCaptionLabelSeparator{custom}{ -- }
\captionsetup[figure]{name=Рисунок, labelsep=custom, font={onehalfspacing}, justification=centering}
\usepackage{ragged2e}
\justifying
\setlength\parindent{1.25cm}
\sloppy
\usepackage{indentfirst}
\author{В.Д. Панков}
\date{\today}
\title{МДК.02.02 Инструментальные средства разработки программного обеспечения}
\hypersetup{
 pdfauthor={В.Д. Панков},
 pdftitle={МДК.02.02 Инструментальные средства разработки программного обеспечения},
 pdfkeywords={},
 pdfsubject={У.С. Опалева},
 pdfcreator={Emacs 28.2 (Org mode 9.5.5)}, 
 pdflang={Russian}}
\begin{document}

\begin{titlepage}

\centering{ГУАП}

\vspace{32pt}

\centering{ФАКУЛЬТЕТ СРЕДНЕГО ПРОФЕССИОНАЛЬНОГО ОБРАЗОВАНИЯ}

\vspace{60pt}

\raggedright{ОТЧЕТ \\
ЗАЩИЩЕН С ОЦЕНКОЙ}
\vspace{14pt}

\raggedright{ПРЕПОДАВАТЕЛЬ}

\vspace{12pt}

\begin{tabularx}{\textwidth}{ >{\centering\arraybackslash}X >{\centering\arraybackslash}X >{\centering\arraybackslash}X }
	 преподаватель & & У.С. Опалева \\ 
	 \hrulefill & \hrulefill & \hrulefill \\ 
\footnotesize{должность, уч. степень, звание} & \footnotesize{подпись, дата} & \footnotesize{инициалы, фамилия} \\ 
\end{tabularx} 
 
\vspace{48pt} 

\centering{ОТЧЕТЫ О ЛАБОРАТОРНЫХ РАБОТАХ} 

\vspace{76pt} 

\centering{По дисциплине: МДК.02.02 Инструментальные средства разработки программного обеспечения} 

\vspace*{\fill} 

\raggedright{РАБОТУ ВЫПОЛНИЛ} 

\vspace{10pt} 

\begin{tabularx}{\textwidth}{>{\raggedright\arraybackslash}X  >{\centering\arraybackslash}X >{\centering\arraybackslash}X >{\centering\arraybackslash}X }
	 СТУДЕНТ ГР. № & 021к & & В.Д. Панков \\ 
	 & \hrulefill & \hrulefill & \hrulefill \\ 
	 &  & \footnotesize{подпись, дата} & \footnotesize{инициалы, фамилия} \\ 
\end{tabularx} 
 
\vspace*{\fill} 

\centering{Санкт-Петербург \the\year} 

\end{titlepage}

\tableofcontents \clearpage
\linespread{1.25}

\section{Лабораторная работа № 1}
\label{sec:org7c47a5d}

Создание консольного приложения в IDE Visual Studio. Работа с системой контроля версий

Цель: изучение IDE Visual Studio, создание простейшего приложения на Python, подключение Git.


​1. Добавил в HelloApp.py

\begin{minted}[breaklines=true,float=t,breakanywhere=true,fontsize=\footnotesize]{python}
print("Hello Python from Visual Studio!")
\end{minted}

​2. Создал Git-репозиторий с помощью PyCharm

\begin{center}
\includegraphics[width=.9\linewidth]{Hello/images/20230222-094130_screenshot.png}
\end{center}


\begin{center}
\includegraphics[width=.9\linewidth]{Hello/images/20230222-094147_screenshot.png}
\end{center}

​3. Добавил файл в Stage

\begin{figure}[htbp]
\centering
\includegraphics[width=.9\linewidth]{Hello/images/20230222-094325_screenshot.png}
\caption{Добавление файла в Stage в Pycharm}
\end{figure}

​4. Создал commit изменений

\begin{figure}[htbp]
\centering
\includegraphics[width=.9\linewidth]{Hello/images/20230222-094612_screenshot.png}
\caption{Демонстрация commit}
\end{figure}

\clearpage

​4. Добавление кода из прошлых заданий по Python

\begin{minted}[breaklines=true,float=t,breakanywhere=true,fontsize=\footnotesize]{python}
print("Hello Python from Visual Studio!")
s: str = "*" * 30
print(s)
print("New project")
print("")

import cProfile
import re

r = re.compile("\\d\\S")
cProfile.run("""[r.findall("sdfdsfD, 1d, 7f") for i in range(1000000)]""")

\end{minted}

\begin{figure}[htbp]
\centering
\includegraphics[width=.9\linewidth]{Hello/images/20230222-103001_screenshot.png}
\caption{Создание нового commit}
\end{figure}


​5. Демонстрация git-log

\begin{center}
\includegraphics[width=.9\linewidth]{Hello/images/20230309-113116_screenshot.png}
\end{center}

\clearpage

​6. Push репозитория на GitHub

\begin{figure}[htbp]
\centering
\includegraphics[width=.9\linewidth]{Hello/images/20230222-094932_screenshot.png}
\caption{Создание репозитория на GitHub}
\end{figure}

\begin{figure}[htbp]
\centering
\includegraphics[width=.9\linewidth]{Hello/images/20230222-095050_screenshot.png}
\caption{Получил ссылку}
\end{figure}

\begin{figure}[htbp]
\centering
\includegraphics[width=.9\linewidth]{Hello/images/20230222-095124_screenshot.png}
\caption{Добавление remote репозитория}
\end{figure}

\begin{figure}[htbp]
\centering
\includegraphics[width=.9\linewidth]{Hello/images/20230222-095207_screenshot.png}
\caption{После добавления remote}
\end{figure}


\begin{figure}[htbp]
\centering
\includegraphics[width=.9\linewidth]{Hello/images/20230222-095249_screenshot.png}
\caption{Процесс push}
\end{figure}


\begin{figure}[htbp]
\centering
\includegraphics[width=.9\linewidth]{Hello/images/20230222-104021_screenshot.png}
\caption{Изменения репозитория}
\end{figure}

\clearpage

​7. Так-как я использую Pycharm, то он не генерирует .gitignore, поэтому самостоятельно был создан .gitignore, который у меня исключает папки:
\begin{itemize}
\item .idea - файлы Jetbrains, которые описывают проект
\item venv - локальный интерпретатор
\end{itemize}

\begin{figure}[htbp]
\centering
\includegraphics[width=.9\linewidth]{Hello/images/20230309-114146_screenshot.png}
\caption{Демонстрация .gitignore}
\end{figure}

\clearpage

​8. Создать новый проект в VS.
Написать код генерирующий список элементов случайным образом из диапазона
от 5 до № по журналу * 100 (число элементов № по журналу + 10).
Выполнить коммит (содержание должно соответствовать задаче).
Оформить код в виде функции, вызвав её с указанным числом элементов.
Добавить коммит. Запушить на GitHub.
На веб-сервисе создать файл README с описанием задачи,
перечнем, включающим среду и язык реализации,
используемые библиотеки, фамилию разработчика.


Код генератора списка:
\begin{minted}[breaklines=true,float=t,breakanywhere=true,fontsize=\footnotesize]{python}
from random import randrange

print([randrange(5, 12 * 100) for i in range(22)])
\end{minted}

Код генератора списка в функции:
\begin{minted}[breaklines=true,float=t,breakanywhere=true,fontsize=\footnotesize]{python}
from random import randrange
from typing import List


def my_random_list(n: int) -> List[int]:
	"""Генерация списка со случайными числами"""
	return [randrange(5, 12 * 100) for i in range(n)]


if __name__ == '__main__':
	print(my_random_list(22))

\end{minted}

Текст README.org файла:
\begin{minted}[breaklines=true,float=t,breakanywhere=true,fontsize=\footnotesize]{org}
#+TITLE: Генератор списка случайных чисел
#+AUTHOR: Панков Василий
#+OPTIONS: toc:nil

Реализовал функцию, которая генерирует список размером n,
элементы которого случайные числа от 5 до 1200.

- Среда: Jetbrains PyCharm
- Язык: Python
- Использованные библиотеки
  - random - для генерации случайных чисел
  - typing - для статической типизации
- Автор: Панков В. Д.

\end{minted}



\begin{figure}[htbp]
\centering
\includegraphics[width=.9\linewidth]{Hello/images/20230309-115518_screenshot.png}
\caption{Первый коммит}
\end{figure}


\begin{figure}[htbp]
\centering
\includegraphics[width=.9\linewidth]{Hello/images/20230309-115559_screenshot.png}
\caption{Второй коммит}
\end{figure}

\begin{figure}[htbp]
\centering
\includegraphics[width=.9\linewidth]{Hello/images/20230309-115641_screenshot.png}
\caption{Git-log}
\end{figure}


\begin{figure}[htbp]
\centering
\includegraphics[width=.9\linewidth]{Hello/images/20230309-115753_screenshot.png}
\caption{Репозиторий на GitHub после двух commit'ов}
\end{figure}


\begin{figure}[htbp]
\centering
\includegraphics[width=.9\linewidth]{Hello/images/20230222-101436_screenshot.png}
\caption{Создание README файла}
\end{figure}

=\begin{figure}[htbp]
\centering
\includegraphics[width=.9\linewidth]{Hello/images/20230309-120532_screenshot.png}
\caption{Итоговый репозиторий на GitHub}
\end{figure}


\clearpage

Контрольные вопросы:

\begin{enumerate}
\item Что такое система контроля версий Git? Требуется ли её установка при работе с VS? 

Git - это система контроля версий,
которая позволяет отслеживать изменения
в исходном коде проекта и управлять ими.
Git позволяет хранить историю изменений,
возвращаться к предыдущим версиям кода, вносить изменения параллельно,
объединять изменения и многое другое.

Для Visual Studio требуется предустановленный Git.

\item Какие основные возможности предоставляет Git в среде VS? 

В среде Visual Studio Git предоставляет возможности по созданию, клонированию, управлению и синхронизации репозиториев Git. Основные возможности Git в среде Visual Studio:
\begin{itemize}
\item Создание новых репозиториев Git

\item Клонирование существующих репозиториев Git

\item Управление изменениями исходного кода в репозитории Git

\item Отслеживание изменений, внесенных другими участниками проекта

\item Работа с ветками и слияние изменений

\item Отправка изменений в удаленный репозиторий и получение изменений из удаленного репозитория.
\end{itemize}

\item Что из настроек Git является обязательным при работе с удалённым репозиторием? 

\begin{itemize}
\item Указание удаленного репозитория, куда будут отправляться изменения.

\item Настройка локальной ветки для отслеживания удаленной ветки.

\item Аутентификация на удаленном репозитории.
\end{itemize}

\item Какую систему защиты и сертификации данных использует Git по умолчанию?

Git может использовать три различных протокола для передачи данных:
Local, HTTP, Secure Shell (SSH).

При хранении файлов на компьютере используется Local.

Для передачи используют: HTTP или Smart HTTP(более умная верисия HTTP).

SSH используется, если он используется сервером.

\item Можно ли вернуться к прежней версии файла с помощью Git? Каким образом? 

Да, можно вернуться к прежней версии файла с помощью Git.
Для этого необходимо использовать команду "git checkout"
с указанием хэша коммита или имени ветки, на которую нужно переключиться.

\item Что нужно сделать, если требуется изменить сообщение последнего коммита?

Если требуется изменить сообщение последнего коммита,
можно использовать команду "git commit --amend".
Она позволяет изменить сообщение последнего коммита
или добавить изменения в него.

\item Как называется главная ветвь разработки? Можно ли её переименовать?

Главная ветвь разработки называется "master". В Git версии 2.28.0 и выше главная ветвь была переименована в "main".
Да, её можно переименовать с помощью команды "git branch -m <old\textsubscript{branch}\textsubscript{name}> <new\textsubscript{branch}\textsubscript{name}>".

\item Зачем нужен файл .gitignore и каким образом он создаётся?

Файл .gitignore предназначен для указания Git файлов и папок,
которые не должны быть добавлены в репозиторий.
Он позволяет исключить файлы и папки, которые не нужны в репозитории,
такие как временные файлы, конфигурационные файлы,
файлы логов, файлы бинарных данных и многое другое.

Файл .gitignore создается в корневой папке проекта.
Он может содержать шаблоны для исключения файлов и папок.
Шаблоны могут включать имя файла или папки, а также использовать
символы подстановки, такие как *, ?, [ ], \{ \}, и многое другое.
Файл .gitignore можно создать вручную или с помощью специальных инструментов,
таких как Visual Studio или Git Extensions.
\end{enumerate}
\end{document}
